\documentclass{jsbook}
\usepackage{amssymb,amsmath}

\newcommand{\cxx}{\textrm{C}\texttt{++}}
\newcommand{\keyword}[1]{\emph{#1}}

\newcommand{\zero}{\mathbf{0}}
\newcommand{\Zero}{\mathbf{O}}
\newcommand{\one}{\mathbf{1}}
\newcommand{\One}{\mathbf{I}}
\newcommand{\im}{\mathbf{i}}
\newcommand{\ii}{\mathbf{i}}
\newcommand{\jj}{\mathbf{j}}
\newcommand{\kk}{\mathbf{k}}
\newcommand{\abs}[1]{\|{#1}\|}
\newcommand{\norm}[1]{\|{#1}\|}
\newcommand{\bvec}[1]{\boldsymbol{#1}}

\title{クォータニオン入門加筆}
\author{金谷一朗}

\begin{document}
\setlength{\baselineskip}{17pt}
\setcounter{chapter}{-1}

\maketitle
\tableofcontents

\chapter{オリジナル版の内容のまとめ}

\section{実数・複素数・クォータニオン --- 数}

\subsection{実数}

\cxx 言語ではdouble型に単項プラス,単項マイナス,和,差,積,商の6個の演算子が定義されている.
これを「double型は\keyword{数としてのインタフェース}を持つ」と言う.

数としてのインタフェースは実際には次のリストに集約される.
\begin{description}
\item[和の演算子] $a+b$の$+$演算子.
\item[零元(ゼロ,和の単位元)] $\zero+a=a+\zero=a$であるような$\zero.$
\item[負元(和の逆元)] $a$に対して$-a+a=\zero$となるような$-a.$
\item[積の演算子] $a\cdot b$の$\cdot$演算子.普通は省略される.
\item[単位元(イチ)] $\one a=a\one=a$であるような$\one.$
\item[逆元] $a$に対して$a^{-1}a=1$であるような$a^{-1}.$
\end{description}
\cxx 言語のdouble型の元になっている\keyword{実数}は上述のインタフェースを持つ.上述の6個のインタフェースは
\begin{description}
\item[和] 演算子,単位元,逆元
\item[積] 演算子,単位元,逆元
\end{description}
という3個ずつのインタフェースに分類できる.

和と積にはそれぞれ次の関係がある.
\begin{align}
abc&=(ab)c\\
  &=a(bc)\\
a+b+c&=(a+b)+c\\
  &=a+(b+c)
\end{align}
これを\keyword{結合則}と呼ぶ.

和と積が混在した場合は常に積が優先される.
\begin{equation}
ab+c=(ab)+c
\end{equation}
和と積の間には次の関係が成り立つ.
\begin{align}
a(b+c)&=ab+ac\\
(a+b)c&=ac+bc
\end{align}
これを\keyword{分配則}と呼ぶ.

零元(ゼロ,和の単位元)と任意の元との積は常に零元である.
\begin{equation}
\zero a=a\zero =\zero
\end{equation}

\subsection{複素数}

実数に限らず,\keyword{複素数}も上述の6個のインタフェース,結合則,分配則に従う.
複素数とは実数単位$\one$の時数倍と虚数単位$\im$の実数倍との和である.
$a,b$を実数とすると,$\alpha=\one a+\im b$が複素数の一般形である.

虚数単位は次の性質を持つ.
\begin{equation}
\im^2=-\one
\end{equation}

複素数は数としてのインタフェースに加えて次のインタフェースを持つ.
\begin{description}
\item[共役複素数] ある複素数$\alpha$が$\alpha=\one a+\im b$であるとき$\alpha^*\equiv\one a-\im b$なる$\alpha^*$を$\alpha$の共役複素数と呼ぶ.
\item[複素数のノルム] ある複素数$\alpha$について,
\begin{equation}
\norm{\alpha}\equiv\sqrt{\alpha^*\alpha}
\end{equation}
を$\alpha$のノルムと呼ぶ.
ノルムは「大きさ」という概念に近い.
\end{description}

複素数$\alpha$の逆数(逆複素数)$\alpha^{-1}$は次のように求めることが出来る.
\begin{equation}
\alpha^{-1}=\frac{\alpha^*}{\norm{\alpha}^2}
\end{equation}

\subsection{クォータニオン}

$Q=\one a+\ii b+\jj c+\kk d$なる数$Q$を\keyword{クォータニオン}(四元数)と呼ぶ.
ただし $\ii,\jj,\kk$ はそれぞれクォータニオン単位であって,
\begin{equation}
\ii^2=\jj^2=\kk^2=\ii\jj\kk=-\one,\,
\ii\jj=-\jj\ii=\kk,\,
\jj\kk=-\kk\jj=\ii,\,
\kk\ii=-\ii\kk=\jj
\end{equation}
であるとする.

クォータニオンは数としてのインタフェースに加えて次のインタフェースを持つ.
\begin{description}
\item[共役クォータニオン] あるクォータニオン$Q$が$Q=\one a+\ii b+\jj c+\kk d$であるとき$Q^*\equiv\one a-\ii b-\jj c-\kk d$なる$Q^*$を$Q$の共役クォータニオンと呼ぶ.
\item[クォータニオンのノルム] あるクォータニオン$Q$について,
\begin{equation}\norm{Q}\equiv\sqrt{Q^*Q}
\end{equation}
を$Q$のノルムと呼ぶ.
ノルムは「大きさ」という概念に近い.
\end{description}

クォータニオン$Q$の逆数(逆クォータニオン)$Q^{-1}$は次のように求めることが出来る.
\begin{equation}
Q^{-1}=\frac{Q^*}{\norm{Q}^2}
\end{equation}

\section{行列 --- もうひとつの数}

\subsection{連立線形方程式と行列}

未知数$x$に関する線形方程式
\begin{equation}
ax+b=\zero
\end{equation}
の解は$x=-a^{-1}b$である.

未知数 $x_1,x_2$ に関する連立線形方程式
\begin{align}
a_{1,1}x_1+a_{1,2}x_2+b_1&=\zero\\
a_{2,1}x_1+a_{2,2}x_2+b_2&=\zero
\end{align}
の解について,新たな記号を発明して
\begin{equation}
\begin{bmatrix}a_{1,1}&a_{1,2}\\a_{2,1}&a_{2,2}\end{bmatrix}\begin{bmatrix}x_1\\x_2\end{bmatrix}+\begin{bmatrix}b_1\\b_2\end{bmatrix}=\begin{bmatrix}\zero\\\zero\end{bmatrix}
\end{equation}
と書き直し,
\begin{equation}
A\equiv\begin{bmatrix}a_{1,1}&a_{1,2}\\a_{2,1}&a_{2,2}\end{bmatrix},\,
X\equiv\begin{bmatrix}x_1\\x_2\end{bmatrix},\,
B\equiv\begin{bmatrix}b_1\\b_2\end{bmatrix},\,
\Zero\equiv\begin{bmatrix}\zero\\\zero\end{bmatrix}
\end{equation}
とすると,未知数 $x_1,x_2$ に関する連立線形方程式は
\begin{equation}
AX+B=\Zero
\end{equation}
と書け,シンプルで美しく見える.
演算規則をうまく調整すると,上述の連立線形方程式の解は$X=-A^{-1}B$と書ける.
このようにして作った $A,B,X,\Zero$ を\keyword{行列}と呼ぶ.

\subsection{正方行列}

各要素が実数からなり,行と列の大きさが等しい行列を\keyword{実正方行列}と呼ぶ.実正方行列を$A$とすると次のように書ける.
\begin{equation}
A=\begin{bmatrix}a_{11}&a_{12}&\dots&a_{1j}&\dots&a_{1n}\\
  a_{21}&a_{22}\\
  \vdots&&\ddots\\
  a_{i1}&&&a_{ij}\\
  \vdots&&&&\ddots\\
  a_{n1}&&&&&a_{nn}\end{bmatrix}
\end{equation}
そこで実正方行列$A$は,その要素と添字を使って$[a_{ij}]$と書くこともある.

実正方行列には
\begin{itemize}
\item 和
\item 零元
\item 負元
\end{itemize}
が定義されている.行列$[a_{ij}]$と行列$[b_{ij}]$の和は
\begin{equation}
[a_{ij}]+[b_{ij}]\equiv[a_{ij}+b_{ij}]
\end{equation}
であり,行列の零元(ゼロ行列)$\Zero$はすべての要素が$\zero$であるような行列である.

行列$[a_{ij}]$と行列$[b_{ij}]$の積も定義されており
\begin{equation}
[a_{ij}][b_{ij}]\equiv\sum_{k=1}^n[a_{ik}b_{kj}]
\end{equation}
である.この定義から,積の単位元(単位行列)$\One$は
\begin{equation}
\One\equiv\begin{bmatrix}1&0&\dots\\
  0&1\\
  \vdots&&\ddots\end{bmatrix}
\end{equation}
でなければならないことがわかる.単位行列$\One$は$[\delta_{ij}]$とも書く.デルタ記号を使うのは歴史的理由である.

\subsection{直交行列とユニタリ行列}

行列$[a_{ij}]$に対して,行と列を入れ替えた$[a_{ji}]$は元の行列の\keyword{転置行列}と呼ばれる.
転置行列は
\begin{equation}
[a_{ij}]^t\equiv[a_{ji}]
\end{equation}
のような記号を使って表す.もし
\begin{equation}
[a_{ij}]^t=[a_{ij}]
\end{equation}
であるならば,行列$[a_{ij}]$は\keyword{対称行列}である.もし
\begin{equation}
[a_{ij}]^t=-[a_{ij}]
\end{equation}
であるならば,行列$[a_{ij}]$は\keyword{反対称行列}である.

実数の代わりに複素数を用いた正方行列を\keyword{複素正方行列}と呼ぶ.
いま複素正方行列を$[\alpha_{ij}]$で表すとき,その共役と転置を行った$[\alpha^*_{ji}]$を\keyword{共役転置行列}と呼ぶ.
共役転置行列を作る操作には特別な記号が割り当てられており,次のように表す.
\begin{equation}
[\alpha_{ij}]^\dagger\equiv[\alpha^*_{ji}]
\end{equation}

もし
\begin{equation}
[a_{ij}]^\dagger=[a_{ij}]
\end{equation}
であるならば,行列$[\alpha_{ij}]$は\keyword{エルミート行列}である.もし
\begin{equation}
[a_{ij}]^\dagger=-[a_{ij}]
\end{equation}
であるならば,行列$[\alpha_{ij}]$は\keyword{反エルミート行列}である.

実正方行列$A$について,もし
\begin{equation}
A^tA=\One
\end{equation}
であるならば,行列$A$は\keyword{直交行列}である.複素正方行列$A$について,もし
\begin{equation}
A^\dagger A=\One
\end{equation}
であるならば,行列$A$は\keyword{ユニタリ行列}である.

% $\det A$

\section{行列による2次元の回転と内積}

\subsection{ベクトル}

ベクトルには
\begin{itemize}
\item 和
\item 零元(ゼロベクトル)
\item 負元(逆ベクトル)
\end{itemize}
がある.またベクトルは実数倍が出来る.

ベクトル$\bvec{p}$のノルム$\norm{\bvec{p}}$という量を定義できる.
ノルムの定義は複数あるが,最もよく用いられているものは,ベクトルをユークリッド空間における位置とみなし,その位置の原点からの距離とする定義である.

二つのベクトル $\bvec{p},\bvec{q}$ の間に\keyword{内積}という演算が定義できる.
内積は $\langle\bvec{p},\bvec{q}\rangle$ で表す.
ベクトルをユークリッド空間における位置とみなしたとき,二つのベクトルのなす角度を$\theta$として,
\begin{equation}
\langle\bvec{p},\bvec{q}\rangle\equiv\norm{\bvec{p}}\norm{\bvec{q}}\cos\theta
\end{equation}
と定義するのが,最も一般的な内積の定義である.
この定義に従えば,ベクトル$\bvec{p}$のノルム$\norm{\bvec{p}}$は
\begin{equation}
\norm{\bvec{p}}=\sqrt{\langle\bvec{p},\bvec{p}\rangle}
\end{equation}
である.

幾何学的な座標系を導入すると便利なことが多々ある.
座標系を表すベクトルを\keyword{基底ベクトル}と呼ぶ.
基底ベクトルとしていま $\bvec{e}_1,\bvec{e}_2$ があるとする.

ベクトル$\bvec{p}$の\keyword{成分}を $p_1,p_2$ で表すと,
\begin{equation}
p_\mu=\langle\bvec{p},\bvec{e}_\mu\rangle
\end{equation}
である.ただし$\mu$は $1,2$ である.ベクトルは成分と基底ベクトルから次のように合成できる.
\begin{equation}
\bvec{p}=\sum_{\mu=1}^2p_\mu\bvec{e}_\mu
\end{equation}

基底ベクトルの組として\keyword{正規直交系}を選ぶとは
\begin{gather}
\norm{\bvec{e}_1}=\norm{\bvec{e}_2}=1\\
\langle\bvec{e}_1,\bvec{e}_2\rangle=0
\end{gather}
を満たすような $\bvec{e}_1,\bvec{e}_2$ を選ぶということである.一般には
\begin{equation}
\langle\bvec{e}_\mu,\bvec{e}_\nu\rangle=\delta_{\mu\nu}
\end{equation}
と書くことが多い.

\subsection{ベクトルの回転}

ベクトル$\bvec{p}$の正規直交系での成分 $p_1,p_2$ を行列風に
\begin{equation}
\begin{bmatrix}p_1\\p_2\end{bmatrix}
\end{equation}
と書くと便利なことがある.
ベクトル$\bvec{p}$で表される位置(これを今後$\vec{P}$としよう)を原点まわりに$\theta$回転させた
位置(これは$\vec{P'}$とする)のベクトル$\bvec{p}'$の成分は次のように計算出来る.
\begin{equation}
\begin{bmatrix}p'_1\\p'_2\end{bmatrix}=\begin{bmatrix}\cos\theta&-\sin\theta\\\sin\theta&\cos\theta\end{bmatrix}\begin{bmatrix}p_1\\p_2\end{bmatrix}
\end{equation}
ここで行列
\begin{equation}
T(\theta)\equiv\begin{bmatrix}\cos\theta&-\sin\theta\\\sin\theta&\cos\theta\end{bmatrix}
\end{equation}
を導入し,ベクトルと行列を意図的に混同すると
\begin{equation}
\bvec{p}'=T(\theta)\bvec{p}
\end{equation}
という簡潔な式が得られる.ここで行列だとか成分だとかを一切忘れて,ベクトル$\bvec{p}$に作用するものとして$T(\theta)$を捉える.
この$T(\theta)$は\keyword{作用素}と呼ばれる.

\section{複素数による2次元の回転}

正規直交系の基底ベクトルとは
\begin{equation}
\langle\bvec{e}_\mu,\bvec{e}_\nu\rangle=\delta_{\mu\nu}
\end{equation}
を満たしてさえいればよい.もし内積の定義を都合よく選べば
\begin{equation}
\bvec{e}_1=\one,\,\bvec{e}_2=\im
\end{equation}
なる座標系を作ることが出来る.実際この座標系は\keyword{複素座標系}またはガウス座標系と呼ばれる.

複素座標系では,内積 $\langle\alpha,\beta\rangle$ を次のように定義する.
\begin{equation}
\langle\alpha,\beta\rangle\equiv\alpha^*\beta
\end{equation}


\section{行列による3次元の回転と外積}
\section{クォータニオンによる3次元の回転}
\section{テンソルとスピノール}

\begin{enumerate}
\item 実数・複素数・クォータニオン --- 数
\item 行列 --- もうひとつの数
\item 行列による2次元の回転と内積
\item 複素数による2次元の回転
\item 行列による3次元の回転と外積
\item クォータニオンによる3次元の回転
\item テンソルとスピノール
\end{enumerate}


\chapter{群について}

\chapter{リー群(リー代数)について}

\chapter{束について}


\end{document}