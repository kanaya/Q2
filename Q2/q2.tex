\documentclass{jsbook}
\usepackage{amssymb,amsmath}

\newcommand{\cxx}{\textrm{C}\texttt{++}}
\newcommand{\cliteral}[1]{\texttt{#1}}
\newcommand{\ckeyword}[1]{\textbf{#1}}
\newcommand{\cid}[1]{\textit{#1}}

\newcommand{\keyword}[1]{\emph{#1}}

\newcommand{\bvec}[1]{\boldsymbol{#1}}
\newcommand{\bop}[1]{\boldsymbol{#1}}

\newcommand{\ii}{I}%{\mathbf{i}}
\newcommand{\jj}{J}%{\mathbf{j}}
\newcommand{\kk}{K}%{\mathbf{k}}

\newcommand{\Zero}{0}%{\mathbf{O}}
\newcommand{\One}{1}%{\mathbf{1}}

\newcommand{\im}{i}%{\mathbf{i}}
\newcommand{\I}{I}%{\mathbf{I}}

\newcommand{\ee}{\mathit{e}}
\newcommand{\abs}[1]{\|{#1}\|}
\newcommand{\norm}[1]{\|{#1}\|}
\newcommand{\ve}{\bvec{e}}

\title{クォータニオン入門加筆}
\author{金谷一朗}

\begin{document}
\setlength{\baselineskip}{17pt}

\maketitle
\tableofcontents

\setcounter{chapter}{-1}
\chapter{オリジナル版の内容のまとめ}

\setcounter{section}{-1}
\section{記号一覧}

\begin{description}
\item[実数] 実数は小文字のローマ文字を使う.例: $a,b,c.$
\item[複素数] 複素数は小文字のギリシャ文字を使う.例: $\alpha,\beta,\gamma.$
\item[行列] 行列は大文字を使う.例: $A,B,C.$
\item[クォータニオン] クォータニオンは大文字のギリシャ文字を使う.例: $\varPhi,\varPsi,\varSigma.$
\item[ベクトル] ベクトルは太字を使う.
ベクトルの表現として複素数,行列,クォータニオンを使うことがあるが,全て小文字のローマ文字を使う.
例: $\bvec{a},\bvec{b},\bvec{c}.$
\item[スピノール] スピノールは太字を使い,全て小文字のギリシャ文字を使う.
例: $\bvec{\alpha},\bvec{\beta},\bvec{\gamma}.$
\item[作用素] ベクトルに対する作用素は太字を使う.
作用素の表現として複素数,行列,クォータニオンを使うことがあるが,全て大文字のローマ文字を使う.
例: $\bop{A},\bop{B},\bop{C}.$
\item[単位] 虚数単位は$\im$で表す. クォータニオン単位は $\ii,\jj,\kk$ で表す.
単位行列は$\One$で,零行列は$\Zero$で表す.
\item[基底ベクトル] 基底ベクトルは$\ve$で表す.
\item[添字] 添字は $i,j,k$ を用いる.
\item[その他] ネイピア数は$e$で表す.クロネッカーのデルタは$\delta$で表す.
円周率は$\pi$で表す.パウリ行列は$\sigma$で表す.
角度にはギリシャ文字を使う.
\end{description}

\section{実数・複素数・クォータニオン --- 数}

\subsection{実数}

\cxx 言語では\ckeyword{double}型に単項プラス,単項マイナス,和,差,積,商の6個の演算子が定義されている.
これを「\ckeyword{double}型は\keyword{数としてのインタフェース}を持つ」と言う.

数としてのインタフェースは実際には次のリストに集約される.
\begin{description}
\item[和の演算子] $a+b$の$+$演算子.\cxx 言語の和演算子.
\item[零元(ゼロ,和の単位元)] $0+a=a+0=a$であるような$0$.\cxx 言語のリテラル\cliteral{0}.
\item[負元(和の逆元)] $a$に対して$-a+a=0$となるような$-a$.\cxx 言語の単項マイナス.
\item[積の演算子] $a\cdot b$の$\cdot$演算子.普通は省略される.\cxx 言語の積演算子.
\item[単位元(イチ)] $1a=a1=a$であるような$1$.\cxx 言語のリテラル\cliteral{1.0}.
\item[逆元] $a$に対して$a^{-1}a=1$であるような$a^{-1}$.\cxx 言語ではデフォルトで用意されていないが
ラムダ式 \texttt{[]}(\ckeyword{double} \cid{x}) \{ \ckeyword{return} \cliteral{1.0}/\cid{x}; \} を用いて容易に実装可能である.
\end{description}
\cxx 言語のdouble型の元になっている\keyword{実数}は上述のインタフェースを持つ.上述の6個のインタフェースは
\begin{description}
\item[和] 演算子,単位元,逆元
\item[積] 演算子,単位元,逆元
\end{description}
という3個ずつのインタフェースに分類できる.

和と積にはそれぞれ次の関係がある.
\begin{align}
abc&=(ab)c\\
  &=a(bc)\\
a+b+c&=(a+b)+c\\
  &=a+(b+c)
\end{align}
これを\keyword{結合則}と呼ぶ.

和と積が混在した場合は常に積が優先される.
\begin{equation}
ab+c=(ab)+c
\end{equation}
和と積の間には次の関係が成り立つ.
\begin{align}
a(b+c)&=ab+ac\\
(a+b)c&=ac+bc
\end{align}
これを\keyword{分配則}と呼ぶ.

零元(ゼロ,和の単位元)と任意の元との積は常に零元である.
\begin{equation}
0 a=a0 =0
\end{equation}

\subsection{複素数}

実数に限らず,\keyword{複素数}も上述の6個のインタフェース,結合則,分配則に従う.
複素数とは実数単位$1$の時数倍と虚数単位$\im$の実数倍との和である.
$a,b$を実数とすると,$\alpha=1 a+\im b$が複素数の一般形である.

虚数単位は次の性質を持つ.
\begin{equation}
\im^2=-1
\end{equation}

複素数は数としてのインタフェースに加えて次のインタフェースを持つ.
\begin{description}
\item[共役複素数] ある複素数$\alpha$が$\alpha=1 a+\im b$であるとき$\alpha^*\equiv1 a-\im b$なる$\alpha^*$を$\alpha$の共役複素数と呼ぶ.
\item[複素数のノルム] ある複素数$\alpha$について,
\begin{equation}
\norm{\alpha}\equiv\sqrt{\alpha^*\alpha}
\end{equation}
を$\alpha$のノルムと呼ぶ.
ノルムは「大きさ」という概念に近い.
\end{description}

複素数$\alpha$の逆数(逆複素数)$\alpha^{-1}$は次のように求めることが出来る.
\begin{equation}
\alpha^{-1}=\frac{\alpha^*}{\norm{\alpha}^2}
\end{equation}

\subsection{クォータニオン}

$\varPhi=1 a+\ii b+\jj c+\kk d$なる数$\varPhi$を\keyword{クォータニオン}(四元数)と呼ぶ.
ただし $\ii,\jj,\kk$ はそれぞれクォータニオン単位であって,
\begin{equation}
\ii^2=\jj^2=\kk^2=\ii\jj\kk=-1,\,
\ii\jj=-\jj\ii=\kk,\,
\jj\kk=-\kk\jj=\ii,\,
\kk\ii=-\ii\kk=\jj
\end{equation}
であるとする.

クォータニオンは数としてのインタフェースに加えて次のインタフェースを持つ.
\begin{description}
\item[共役クォータニオン] あるクォータニオン$\varPhi$が$\varPhi=1 a+\ii b+\jj c+\kk d$である
とき$\varPhi^*\equiv1 a-\ii b-\jj c-\kk d$なる$\varPhi^*$を$\varPhi$の共役クォータニオンと呼ぶ.
\item[クォータニオンのノルム] あるクォータニオン$\varPhi$について,
\begin{equation}\norm{\varPhi}\equiv\sqrt{\varPhi^*\varPhi}
\end{equation}
を$\varPhi$のノルムと呼ぶ.
ノルムは「大きさ」という概念に近い.
\end{description}

クォータニオン$\varPhi$の逆数(逆クォータニオン)$\varPhi^{-1}$は次のように求めることが出来る.
\begin{equation}
\varPhi^{-1}=\frac{\varPhi^*}{\norm{\varPhi}^2}
\end{equation}

\section{行列 --- もうひとつの数}

\subsection{連立線形方程式と行列}

未知数$x$に関する線形方程式
\begin{equation}
ax+b=0
\end{equation}
の解は$x=-a^{-1}b$である.

未知数 $x_1,x_2$ に関する連立線形方程式
\begin{align}
a_{1,1}x_1+a_{1,2}x_2+b_1&=0\\
a_{2,1}x_1+a_{2,2}x_2+b_2&=0
\end{align}
の解について,新たな記号を発明して
\begin{equation}
\begin{bmatrix}a_{1,1}&a_{1,2}\\a_{2,1}&a_{2,2}\end{bmatrix}\begin{bmatrix}x_1\\x_2\end{bmatrix}+\begin{bmatrix}b_1\\b_2\end{bmatrix}=\begin{bmatrix}0\\0\end{bmatrix}
\end{equation}
と書き直し,
\begin{equation}
A\equiv\begin{bmatrix}a_{1,1}&a_{1,2}\\a_{2,1}&a_{2,2}\end{bmatrix},\,
X\equiv\begin{bmatrix}x_1\\x_2\end{bmatrix},\,
B\equiv\begin{bmatrix}b_1\\b_2\end{bmatrix},\,
\Zero\equiv\begin{bmatrix}0\\0\end{bmatrix}
\end{equation}
とすると,未知数 $x_1,x_2$ に関する連立線形方程式は
\begin{equation}
AX+B=\Zero
\end{equation}
と書け,シンプルで美しく見える.
演算規則をうまく調整すると,上述の連立線形方程式の解は$X=-A^{-1}B$と書ける.
このようにして作った $A,B,X,\Zero$ を\keyword{行列}と呼ぶ.

行列$A$の逆行列$A^{-1}$が存在するか否かの判定(determinant)に\keyword{行列式}という演算子が使われる.
行列$A$の行列式は $\det A$ または$|A|$と書く.

\subsection{正方行列}

各要素が実数からなり,行と列の大きさが等しい行列を\keyword{実正方行列}と呼ぶ.実正方行列を$A$とすると次のように書ける.
\begin{equation}
A=\begin{bmatrix}a_{11}&a_{12}&\dots&a_{1j}&\dots&a_{1n}\\
  a_{21}&a_{22}\\
  \vdots&&\ddots\\
  a_{i1}&&&a_{ij}\\
  \vdots&&&&\ddots\\
  a_{n1}&&&&&a_{nn}\end{bmatrix}
\end{equation}
そこで実正方行列$A$は,その要素と添字を使って$[a_{ij}]$と書くこともある.

実正方行列には
\begin{itemize}
\item 和
\item 零元
\item 負元
\end{itemize}
が定義されている.行列$[a_{ij}]$と行列$[b_{ij}]$の和は
\begin{equation}
[a_{ij}]+[b_{ij}]\equiv[a_{ij}+b_{ij}]
\end{equation}
であり,行列の零元(ゼロ行列)$\Zero$はすべての要素が$0$であるような行列である.

行列$[a_{ij}]$と行列$[b_{ij}]$の積も定義されており
\begin{equation}
[a_{ij}][b_{ij}]\equiv\sum_{k=1}^n[a_{ik}b_{kj}]
\end{equation}
である.この定義から,積の単位元(単位行列)$\One$は
\begin{equation}
\One\equiv\begin{bmatrix}1&0&\dots&0\\
  0&1\\
  \vdots&&\ddots\\
  0&&&1\end{bmatrix}
\end{equation}
でなければならないことがわかる.単位行列$\One$は$[\delta_{ij}]$とも書く.デルタ記号を使うのは歴史的理由である.

\subsection{直交行列とユニタリ行列}

行列$[a_{ij}]$に対して,行と列を入れ替えた$[a_{ji}]$は元の行列の\keyword{転置行列}と呼ばれる.
転置行列は
\begin{equation}
[a_{ij}]^t\equiv[a_{ji}]
\end{equation}
のような記号を使って表す.もし
\begin{equation}
[a_{ij}]^t=[a_{ij}]
\end{equation}
であるならば,行列$[a_{ij}]$は\keyword{対称行列}である.もし
\begin{equation}
[a_{ij}]^t=-[a_{ij}]
\end{equation}
であるならば,行列$[a_{ij}]$は\keyword{反対称行列}である.

実数の代わりに複素数を用いた正方行列を\keyword{複素正方行列}と呼ぶ.
いま複素正方行列を$[\alpha_{ij}]$で表すとき,その共役と転置を行った$[\alpha^*_{ji}]$を\keyword{共役転置行列}と呼ぶ.
共役転置行列を作る操作には特別な記号が割り当てられており,次のように表す.
\begin{equation}
[\alpha_{ij}]^\dagger\equiv[\alpha^*_{ji}]
\end{equation}

もし
\begin{equation}
[a_{ij}]^\dagger=[a_{ij}]
\end{equation}
であるならば,行列$[\alpha_{ij}]$は\keyword{エルミート行列}である.もし
\begin{equation}
[a_{ij}]^\dagger=-[a_{ij}]
\end{equation}
であるならば,行列$[\alpha_{ij}]$は\keyword{反エルミート行列}である.

実正方行列$A$について,もし
\begin{equation}
A^tA=\One
\end{equation}
であるならば,行列$A$は\keyword{直交行列}である.複素正方行列$A$について,もし
\begin{equation}
A^\dagger A=\One
\end{equation}
であるならば,行列$A$は\keyword{ユニタリ行列}である.

% $\det A$

\section{行列による2次元の回転と内積}

\subsection{ベクトル}

ベクトルには
\begin{itemize}
\item 和
\item 零元(ゼロベクトル)
\item 負元(逆ベクトル)
\end{itemize}
がある.またベクトルは実数倍が出来る.

ベクトル$\bvec{p}$のノルム$\norm{\bvec{p}}$という量を定義できる.
ノルムの定義は複数あるが,最もよく用いられているものは,ベクトルをユークリッド空間における位置とみなし,その位置の原点からの距離とする定義である.

\subsection{内積}

二つのベクトル $\bvec{p},\bvec{q}$ の間に\keyword{内積}という演算が定義できる.
内積は $\langle\bvec{p},\bvec{q}\rangle$ で表す.
ベクトルをユークリッド空間における位置 $\overrightarrow{OP},\overrightarrow{OQ}$ とみなしたとき,
二つのベクトルのなす角度を$\theta$として,
\begin{equation}
\langle\bvec{p},\bvec{q}\rangle\equiv\norm{\bvec{p}}\norm{\bvec{q}}\cos\theta
\end{equation}
と定義するのが,最も一般的な内積の定義である.
この定義に従えば,ベクトル$\bvec{p}$のノルム$\norm{\bvec{p}}$は
\begin{equation}
\norm{\bvec{p}}=\sqrt{\langle\bvec{p},\bvec{p}\rangle}
\end{equation}
である.

幾何学的な座標系を導入すると便利なことが多々ある.
座標系を表すベクトルを\keyword{基底ベクトル}と呼ぶ.
基底ベクトルとしていま $\ve_1,\ve_2$ があるとする.

ベクトル$\bvec{p}$の\keyword{成分}を $p_1,p_2$ で表すと,
\begin{equation}
p_i=\langle\bvec{p},\ve_i\rangle
\end{equation}
である.ただし$i$は $1,2$ である.ベクトルは成分と基底ベクトルから次のように合成できる.
\begin{equation}
\bvec{p}=\sum_{i=1}^2p_i\ve_i
\end{equation}

基底ベクトルの組として\keyword{正規直交系}を選ぶとは
\begin{gather}
\norm{\ve_1}=\norm{\ve_2}=1\\
\langle\ve_1,\ve_2\rangle=0
\end{gather}
を満たすような $\ve_1,\ve_2$ を選ぶということである.一般には
\begin{equation}
\langle\ve_i,\ve_j\rangle=\delta_{ij}
\end{equation}
と書くことが多い.

\subsection{ベクトルの回転}

ベクトル$\bvec{p}$の正規直交系での成分 $p_1,p_2$ を行列風に
\begin{equation}
\begin{bmatrix}p_1\\p_2\end{bmatrix}
\end{equation}
と書くと便利なことがある.
ベクトル$\bvec{p}$で表される位置(これを今後$\overrightarrow{OP}$としよう)を原点まわりに$\theta$回転させた
位置(これは$\overrightarrow{OP'}$とする)のベクトル$\bvec{p}'$の成分は次のように計算出来る.
\begin{equation}
\begin{bmatrix}p'_1\\p'_2\end{bmatrix}=\begin{bmatrix}\cos\theta&-\sin\theta\\\sin\theta&\cos\theta\end{bmatrix}\begin{bmatrix}p_1\\p_2\end{bmatrix}
\end{equation}
証明はオリジナル版を参照.

ここで行列
\begin{equation}
\bop{T}(\theta)\equiv\begin{bmatrix}\cos\theta&-\sin\theta\\\sin\theta&\cos\theta\end{bmatrix}
\end{equation}
を導入し,ベクトルと行列を意図的に混同すると
\begin{equation}
\bvec{p}'=\bop{T}(\theta)\bvec{p}
\end{equation}
という簡潔な式が得られる.ここで行列だとか成分だとかを一切忘れて,ベクトル$\bvec{p}$に作用するものとして$\bop{T}(\theta)$を捉える.
この$\bop{T}(\theta)$は\keyword{作用素}と呼ばれる.

\section{複素数による2次元の回転}

\subsection{複素数で表す2次元ベクトル}

正規直交系の基底ベクトルとは
\begin{equation}
\langle\ve_i,\ve_j\rangle=\delta_{ij}
\end{equation}
を満たしてさえいればよい.もし内積の定義を都合よく選べば
\begin{equation}
\ve_1=1,\,\ve_2=\im
\end{equation}
なる座標系を作ることが出来る.実際この座標系は\keyword{複素座標系}またはガウス座標系と呼ばれる.
ここに内積の定義として
\begin{equation}
\langle\alpha,\beta\rangle\equiv\alpha^*\beta
\end{equation}
を採用した.

\subsection{回転}

複素座標系における回転の作用素$\bop{U}(\theta)$は次の形を取る.
\begin{equation}
\bop{U}(\theta)=\cos\theta+\im\sin\theta
\end{equation}
%
% なぜ?
%

...
\begin{equation}
\bvec{p}'=\bop{U}(\theta)\bvec{p}
\end{equation}

オイラーの公式
\begin{equation}
\exp\im\theta=\cos\theta+\im\sin\theta
\end{equation}
を用いると,回転$\bop{U}(\theta)$は
\begin{equation}
\bop{U}(\theta)=\exp\im\theta
\end{equation}
とさらに簡潔に書ける.

\subsection{ベクトルと行列と複素数の関係}

2次元ベクトルが行列でも複素数でも書けるのは,基底ベクトルの取り方次第だからである.
基底ベクトルに正規直交系を選ぶと便利であった.正規直交系とは基底ベクトル$\bvec{p}_i$が
\begin{equation}
\langle\ve_i,\ve_j\rangle=\delta_{ij}
\end{equation}
でありさえすればよく,内積をうまく定義してやれば自由に基底ベクトルを選べる.

行列スタイルを採用して
\begin{equation}
\ve_1=\begin{bmatrix}1\\0\end{bmatrix},\,
\ve_2=\begin{bmatrix}0\\1\end{bmatrix}
\end{equation}
としても良かったし,複素数スタイルを採用して
\begin{equation}
\ve_1=1,\,
\ve_2=\im
\end{equation}
としても良かった.
どちらかと言えば複素数スタイルのほうが数としてのインタフェースを使えるので優れているとは言える.
そこで数としてのインタフェースを保ちつつ行列も使えないかと考えると
\begin{equation}
\ve_1=\begin{bmatrix}1&0\\0&1\end{bmatrix},\,
\ve_2=\begin{bmatrix}0&-1\\1&0\end{bmatrix}
\end{equation}
という基底ベクトルも良いことに気づくだろう.
この場合$\ve_1$のほうは単位行列$\One$と同じであるので,
もうひとつの$\ve_2$のほうを虚数単位$\im$に対応させて
\begin{equation}
\im'\equiv\begin{bmatrix}0&-1\\1&0\end{bmatrix}
\end{equation}
と名づけても構わない.
%
% それで?
%

\section{行列による3次元の回転と外積}

\subsection{外積}

2次元のユークリッド空間を3次元に拡張するのはわけないことだ.
とりわけ行列スタイルであればほとんど自動的に
\begin{equation}
\ve_1=\begin{bmatrix}1\\0\\0\end{bmatrix},\,
\ve_2=\begin{bmatrix}0\\1\\0\end{bmatrix},\,
\ve_3=\begin{bmatrix}0\\0\\1\end{bmatrix},\,
\end{equation}
を採用すれば良いことがわかる.

ここで,3次元空間で非常にうまくいくトリックを導入する.次に述べる\keyword{外積}という演算を3次元ベクトル同士に定義する.
\begin{equation}
\bvec{r}=\bvec{p}\times\bvec{q}
\end{equation}
ここにベクトル$\bvec{r}$はベクトル$\bvec{p}$および$\bvec{q}$に直交し,そのノルムがベクトル$\bvec{p}$とベクトル$\bvec{q}$の張る平行四辺形に等しいとする.
ベクトル$\bvec{r}$の向きは,右手で直交座標系を作り,ベクトル$\bvec{p}$を右手親指,ベクトル$\bvec{q}$を右手人差し指とした場合,右手中指の方向である.
%
% 直交の定義がまだ
%
定義から,ベクトル$\bvec{p}$とベクトル$\bvec{q}$の角度を$\theta$としたときに
\begin{equation}
\norm{\bvec{r}}=\norm{\bvec{p}}\norm{\bvec{q}}\sin\theta
\end{equation}
である.

外積は成分ごとに計算すると手っ取り早い.
\begin{equation}
\bvec{p}\times\bvec{q}=\begin{bmatrix}p_2q_3-p_3q_2\\p_3q_1-p_1q_3\\p_1q_2-p_2q_1\end{bmatrix}
\end{equation}
少しでもスタイリッシュにしたければ行列式を使うことは出来る.
\begin{equation}
\bvec{p}\times\bvec{q}=\det\begin{bmatrix}\ve_1&p_1&q_1\\\ve_2&p_2&q_2\\\ve_3&p_3&q_3\end{bmatrix}
\end{equation}

三重積
\begin{equation}
\bvec{p}\times\bvec{q}\times\bvec{r}=\bvec{q}\langle\bvec{p},\bvec{r}\rangle-\bvec{r}\langle\bvec{p},\bvec{q}\rangle
\end{equation}
は大切な関係である.

\subsection{回転}

3次元ユークリッド空間の回転を考える.いま3軸まわりの回転だけを考えると,それは2次元の回転と変わらない.
3軸まわりの$\theta$回転を$\bop{T}_3(\theta)$とすると
\begin{equation}
\bop{T}_3(\theta)=\begin{bmatrix}\cos\theta&-\sin\theta&0\\\sin\theta&\cos\theta&0\\0&0&1\end{bmatrix}
\end{equation}
である.同じく2軸まわりは
\begin{equation}
\bop{T}_2(\theta)=\begin{bmatrix}\cos\theta&0&\sin\theta\\0&1&0\\-\sin\theta&0&\cos\theta\end{bmatrix}
\end{equation}
であり,1軸まわりは
\begin{equation}
\bop{T}_1(\theta)=\begin{bmatrix}1&0&0\\0&\cos\theta&-\sin\theta\\0&\sin\theta&\cos\theta\end{bmatrix}
\end{equation}
である.

これらの回転行列のうち,ふたつを組み合わせれば3次元の回転は全て表現できる.

\subsection{もう一つの回転}

回転の計算に外積を使うことも出来る.ベクトル$\bvec{p}$をベクトル$\bvec{r}$まわりに$\theta$回転させたベクトル$\bvec{p}'$は
\begin{equation}
\bvec{p}'=\bvec{p}\cos\theta+\bvec{r}\times\bvec{p}\sin\theta+\bvec{r}\langle\bvec{r},\bvec{p}\rangle(1-\cos\theta)
\end{equation}
である.ただし$\norm{\bvec{r}}=1$を仮定した.証明はオリジナル版を参照.

\section{クォータニオンによる3次元の回転}

\subsection{パウリ行列}

2次元の場合,正規直交系の基底ベクトルとして行列と複素数のどちらも選べた.3次元の場合の複素数に相当する基底ベクトルはあるだろうか.
次の複素行列は3次元の正規直交基底であることが知られている.
\begin{equation}
\sigma_1=\begin{bmatrix}0&1\\1&0\end{bmatrix},\,
\sigma_2=\begin{bmatrix}0&-\im\\\im&0\end{bmatrix},\,
\sigma_3=\begin{bmatrix}1&0\\0&-1\end{bmatrix}
\end{equation}
これらの行列は\keyword{パウリ行列}と呼ばれている.

パウリ行列は様々な良い性質を持つ.
各々の行列の自乗は単位行列になる.
\begin{equation}
{\sigma_1}^2=\One,\,
{\sigma_2}^2=\One,\,
{\sigma_3}^2=\One
\end{equation}

各々の行列の積は,残りの行列になる.
\begin{equation}
\sigma_1\sigma_2=\sigma_3,\,
\sigma_2\sigma_3=\sigma_1,\,
\sigma_3\sigma_1=\sigma_2
\end{equation}
この性質は,すなわち通常の行列積がベクトルの外積として使えることを示す.

内積...

\subsection{クォータニオン}

パウリ行列に一工夫を加えると,クォータニオンが得られる.
\begin{equation}
\varPhi=1a+\im\sigma_3b+\im\sigma_2c+\im\sigma_1d
\end{equation}
なる量$\varPhi$はクォータニオンとしての性質をすべて持つ.また $\im\sigma_3,\im\sigma_2,\im\sigma_1$ はクォータニオン単位の性質を持つ.そこで
\begin{equation}
\ve_1=\im\sigma_3,\,
\ve_2=\im\sigma_2,\,
\ve_3=\im\sigma_1
\end{equation}
を基底ベクトルとして採用しよう.

ベクトル$\bvec{p}$をベクトル$\bvec{r}$まわりに$\theta$回転させる演算子を$U(\bvec{r},\theta)$とする.
回転後のベクトル$\bvec{p}'$は演算子$\bop{U}(\bvec{r},\theta)$を用いて
\begin{equation}
\bvec{p}'=\bop{U}^*(\bvec{r},\theta)\bvec{p}\bop{U}(\bvec{r},\theta)
\end{equation}
のように計算できる.ここに
\begin{equation}
\bop{U}(\theta)=1\cos\frac{\theta}{2}+\bvec{r}\sin\frac{\theta}{2}
\end{equation}
である.証明はオリジナル版にある.

\subsection{球面線形補間}

省略.

\section{テンソルとスピノール}

ベクトル$\bvec{p}$を成分で$p_i$と書いてみる.
回転の演算子$\bop{T}$も成分で$T_{ij}$と書いてみる.
ベクトルの回転は
\begin{equation}
p_j'=\sum_{i=1}^NT_{ij}p_i
\end{equation}
である.行列の書き方を用いると次のように書き直せる.
\begin{equation}
[p_i']=[T_{ij}][p_i]
\end{equation}
または
\begin{equation}
\bvec{p}'=\bop{T}\bvec{p}
\end{equation}

このように変換される$p_i$を1階の\keyword{テンソル}と呼ぶ.
次のように変換されるテンソルもあり,これを2階テンソルと呼ぶ.
\begin{equation}
P_{kl}'=\sum_{i=1}^N\sum_{j=1}^NT_{ik}T_{jl}P_{ik}
\end{equation}
この式を行列を用いて書くと,行列の演算の非対称性から若干の工夫が必要になる.結局
\begin{equation}
[P_{ij}']=[T_{ij}]^t[P_{ij}][T_{ij}]
\end{equation}
または
\begin{equation}
\bvec{P}'=\bop{T}^t\bvec{P}\bop{T}
\end{equation}
となる.

繰り返すと1階テンソルとは
\begin{equation}
\bvec{p}'=\bop{T}\bvec{p}
\end{equation}
と変換される量である.2階テンソルとは
\begin{equation}
\bvec{P}'=\bop{T}^t\bvec{P}\bop{T}
\end{equation}
と変換される量である.

ここで1階テンソルはクォータニオンを使えば
\begin{equation}
\bvec{p}'=\bop{U}^*\bvec{p}\bop{U}
\end{equation}
と書けたことを思い出そう.では
\begin{equation}
\bvec{\psi}'=\bop{U}\bvec{\psi}
\end{equation}
なる量$\bvec{\psi}$はあるだろうか.
この$\bvec{\psi}$こそが\keyword{スピノール}である.

スピノールは行列で表示できる.
\begin{equation}
\bvec{\psi}=\begin{bmatrix}\psi_1\\\psi_2\end{bmatrix}
\end{equation}
\keyword{共役スピノール}を定義しておくと,スピノールの内積が計算しやすい.
\begin{equation}
\bvec{\psi}^*=\begin{bmatrix}-\psi_2&\psi_1\end{bmatrix}
\end{equation}
こうしておけば,スピノールの内積は
\begin{equation}
\langle\bvec{\psi},\bvec{\phi}\rangle=\bvec{\psi}^*\bvec{\phi}
\end{equation}
と演算できて都合が良い.展開すると
\begin{align}
\langle\bvec{\psi},\bvec{\phi}\rangle&=\bvec{\psi}^*\bvec{\phi}\\
  &=\begin{bmatrix}-\psi_2&\psi_1\end{bmatrix}\begin{bmatrix}\phi_1\\\phi_2\end{bmatrix}\\
    &=\psi_1\phi_2-\psi_2\phi_1
\end{align}
であり,この量は回転に対して不変である.

スピノールは$2\pi$回転で符号が入れ替わる.
\begin{equation}
\bvec{\psi}=-U(2\pi)\bvec{\psi}
\end{equation}

スピノールの掛け算の間にパウリ行列を挟むと楽しい.
\begin{equation}
p_i=\bvec{\psi}^*\sigma_i\bvec{\phi}
\end{equation}
このようにして出来た$p_i$は1階3元テンソルとしての変換性を示す.
いま
\begin{equation}
\sigma_0\equiv\One
\end{equation}
を導入すると,
\begin{align}
\langle\bvec{\psi},\bvec{\phi}\rangle&=\bvec{\psi}^*\sigma_0\bvec{\phi}\\
p_i&=\bvec{\psi}^*\sigma_i\bvec{\phi}
\end{align}
であるから$\langle\bvec{\psi},\bvec{\phi}\rangle$と$p_i$をひとつにして,4元のテンソル$p_i$ただし$i=\{0,1,2,3\}$を考えても良い.






\chapter{群について}

\chapter{リー群(リー代数)について}

\chapter{束について}


\end{document}